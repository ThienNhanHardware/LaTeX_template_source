% Gói hỗ trợ màu sắc
\usepackage{xcolor}
\definecolor{TLgreen}{HTML}{00a79d}

% Gói hỗ trợ toán học và định dạng
\usepackage{amsmath}
\usepackage{amsfonts}
\usepackage{listings}
\lstset{
    basicstyle=\ttfamily\small,   % Font monospace nhỏ
    keywordstyle=\color{blue},    % Màu xanh cho từ khóa
    stringstyle=\color{red},      % Màu đỏ cho chuỗi
    commentstyle=\color{gray},    % Màu xám cho comment
    identifierstyle=\color{black},% Màu đen cho tên biến/hàm
    numberstyle=\tiny\color{gray},% Số dòng nhỏ màu xám
    backgroundcolor=\color{lightgray!20}, % Nền xám nhạt
    numbers=left,                 % Hiển thị số dòng
    frame=single,                  % Viền quanh đoạn code
    breaklines=true,               % Tự động xuống dòng
    breakatwhitespace=true         % Chỉ xuống dòng tại khoảng trắng
}


% Gói hỗ trợ hình ảnh và liên kết
\usepackage{graphicx}
\usepackage[hidelinks]{hyperref}
\usepackage{subcaption}
\usepackage{longtable}
\setlength{\parindent}{0pt}
% Hỗ trợ tiếng Việt và Unicode
\usepackage[T5]{fontenc}       % Hỗ trợ mã hóa font
\usepackage[utf8]{inputenc}    % Hỗ trợ Unicode UTF-8

% Change to '[english,vietnamese]' to use Vietnamese labels
\usepackage[vietnamese]{babel}

% Thiết lập lề trang
\usepackage[a4paper, left=0.5in, right=1in, top=1in, bottom=1in]{geometry} % Lề trái nhỏ hơn để cột màu sát viền

% Gói hỗ trợ bố cục nhiều cột
\usepackage{multicol}

\usepackage{tikz}
\usepackage{transparent}
\usetikzlibrary{shadings,fadings}
\usepackage{geometry}
\geometry{top=3cm, bottom=3cm, left=3cm, right=3cm}

\usetikzlibrary{calc} % Include TikZ calc library for coordinate calculations

\usepackage{fancyhdr}  % Gói hỗ trợ chỉnh header và footer
\usepackage{background} % Gói hỗ trợ thiết lập nền
\backgroundsetup{
    scale=1,
    color=TLgreen,
    opacity=1,
    position={8cm,2.1cm},
    angle=0,
    vshift=0cm,
    % hshift=-1in,
    pages=all,
    % contents={\rule{25cm}{1cm}},
    contents={
        \ifnum\value{page}>1
            \rule{25cm}{1cm}
        \fi
    },
}

\pagestyle{fancy}      % Kích hoạt kiểu header/footer fancy
\fancyhf{}             % Xóa header/footer mặc định


% Thiết lập header
\fancyhead[L]{\includegraphics[height = 0.75cm]{figures/logo/START_HERE__GO_ANYWHERE-removebg-preview.png}}      % Chính giữa header (tên chương)
\fancyhead[R]{\docsName \\ \leftmark}   % Góc trái header
% \fancyhead[C]{\thepage}     % Góc phải header (số trang)

% Thiết lập footer
% \fancyfoot[L]{Tác giả}      % Góc trái footer
% \fancyfoot[C]{\today}       % Chính giữa footer (ngày hiện tại)
\fancyfoot[C]{\thepage} % Góc phải footer (số trang)

% Xóa đường kẻ mặc định dưới header
\renewcommand{\headrulewidth}{0.4pt}
\renewcommand{\footrulewidth}{0.4pt}  % Đường kẻ dưới footer

\newcommand{\docsName}{Documentation Name}
\newcommand{\authNameFirst}{Author 1}
\newcommand{\authNameSecond}{Author 2}
\newcommand{\authNameThird}{Author 3}
\newcommand{\authNameFourth}{Author 4}
\newcommand{\authNameFifth}{Author 5}
\newcommand{\cityName}{City}
\newcommand{\orgName}{Organization}

% Cho phép người dùng thay đổi giá trị các biến
\newcommand{\setDocsName}[1]{\renewcommand{\docsName}{#1}}
\newcommand{\setAuthNameFirst}[1]{\renewcommand{\authNameFirst}{#1}}
\newcommand{\setAuthNameSecond}[1]{\renewcommand{\authNameSecond}{#1}}
\newcommand{\setAuthNameThird}[1]{\renewcommand{\authNameThird}{#1}}
\newcommand{\setAuthNameFourth}[1]{\renewcommand{\authNameFourth}{#1}}
\newcommand{\setAuthNameFifth}[1]{\renewcommand{\authNameFifth}{#1}}
\newcommand{\setCityName}[1]{\renewcommand{\cityName}{#1}}
\newcommand{\setOrgName}[1]{\renewcommand{\orgName}{#1}}
