% ==========================
% Gói hỗ trợ màu sắc
% ==========================
\usepackage{xcolor}
\definecolor{TLgreen}{HTML}{00a79d}

% ==========================
% Gói hỗ trợ toán học và định dạng
% ==========================
\usepackage{amsmath}    % Hỗ trợ các công thức toán học
\usepackage{amsfonts}   % Hỗ trợ các font toán học
\usepackage{listings}   % Hỗ trợ hiển thị mã nguồn
\lstset{
    basicstyle=\ttfamily\small,       % Font monospace nhỏ
    keywordstyle=\color{blue},        % Màu xanh cho từ khóa
    stringstyle=\color{red},          % Màu đỏ cho chuỗi
    commentstyle=\color{gray},        % Màu xám cho comment
    identifierstyle=\color{black},    % Màu đen cho tên biến/hàm
    numberstyle=\tiny\color{gray},    % Số dòng nhỏ màu xám
    backgroundcolor=\color{lightgray!20}, % Nền xám nhạt
    numbers=left,                     % Hiển thị số dòng
    frame=single,                     % Viền quanh đoạn code
    breaklines=true,                  % Tự động xuống dòng
    breakatwhitespace=true            % Chỉ xuống dòng tại khoảng trắng
}

% ==========================
% Gói hỗ trợ hình ảnh và liên kết
% ==========================
\usepackage{graphicx}                % Hỗ trợ chèn hình ảnh
\usepackage[hidelinks]{hyperref}     % Hỗ trợ liên kết, ẩn khung viền
\usepackage{subcaption}              % Hỗ trợ chú thích con
\usepackage{longtable}               % Hỗ trợ bảng dài
\setlength{\parindent}{0pt}          % Không thụt đầu dòng
\usepackage{parskip}                 % Thêm khoảng cách giữa các đoạn

% ==========================
% Hỗ trợ tiếng Việt và Unicode
% ==========================
\usepackage[T5]{fontenc}             % Hỗ trợ mã hóa font
\usepackage[utf8]{inputenc}          % Hỗ trợ Unicode UTF-8
\usepackage[english]{babel}          % Ngôn ngữ (vietnamese/english)

% ==========================
% Thiết lập lề trang
% ==========================
\usepackage[a4paper, left=0.5in, right=1in, top=1in, bottom=1in]{geometry} % Lề trang chuẩn
\geometry{top=3cm, bottom=3cm, left=3cm, right=3cm} % Ghi đè lề trang nếu cần

% ==========================
% Gói hỗ trợ bố cục nhiều cột
% ==========================
\usepackage{multicol}

% ==========================
% Gói hỗ trợ đồ họa và nền
% ==========================
\usepackage{tikz}                    % Hỗ trợ đồ họa
\usepackage{transparent}             % Hỗ trợ độ trong suốt
\usetikzlibrary{shadings, fadings}   % Thư viện TikZ cho hiệu ứng
\usetikzlibrary{calc}                % Thư viện TikZ cho tính toán tọa độ

% ==========================
% Header và Footer
% ==========================
\usepackage{fancyhdr}                % Hỗ trợ chỉnh header và footer
\pagestyle{fancy}                    % Kích hoạt kiểu header/footer fancy
\fancyhf{}                           % Xóa header/footer mặc định

% Thiết lập header
\fancyhead[L]{\includegraphics[height=0.75cm]{figures/logo/START_HERE__GO_ANYWHERE-removebg-preview.png}} % Logo góc trái
\fancyhead[R]{\docsName \\ \leftmark} % Tên tài liệu và tên chương góc phải

% Thiết lập footer
\fancyfoot[C]{\thepage}              % Số trang ở giữa footer

% Đường kẻ header/footer
\renewcommand{\headrulewidth}{0.4pt} % Độ dày đường kẻ header
\renewcommand{\footrulewidth}{0.4pt} % Độ dày đường kẻ footer

% ==========================
% Gói hỗ trợ thiết lập nền
% ==========================
\usepackage{background}
\backgroundsetup{
    scale=1,
    color=TLgreen,
    opacity=1,
    position={8cm,2.1cm},
    angle=0,
    vshift=0cm,
    pages=all,
    contents={
        \ifnum\value{page}>1
            \rule{25cm}{1cm}
        \fi
    },
}

% ==========================
% Biến tùy chỉnh
% ==========================
\newcommand{\docsName}{Documentation Name}
\newcommand{\authNameFirst}{Author 1}
\newcommand{\authNameSecond}{Author 2}
\newcommand{\authNameThird}{Author 3}
\newcommand{\authNameFourth}{Author 4}
\newcommand{\authNameFifth}{Author 5}
\newcommand{\cityName}{City}
\newcommand{\orgName}{Organization}

% Cho phép người dùng thay đổi giá trị các biến
\newcommand{\setDocsName}[1]{\renewcommand{\docsName}{#1}}
\newcommand{\setAuthNameFirst}[1]{\renewcommand{\authNameFirst}{#1}}
\newcommand{\setAuthNameSecond}[1]{\renewcommand{\authNameSecond}{#1}}
\newcommand{\setAuthNameThird}[1]{\renewcommand{\authNameThird}{#1}}
\newcommand{\setAuthNameFourth}[1]{\renewcommand{\authNameFourth}{#1}}
\newcommand{\setAuthNameFifth}[1]{\renewcommand{\authNameFifth}{#1}}
\newcommand{\setCityName}[1]{\renewcommand{\cityName}{#1}}
\newcommand{\setOrgName}[1]{\renewcommand{\orgName}{#1}}
