\chapter{S-Parameters}
    \textbf{S-Parameters} là một công cụ có giá trị để tính toán 
    \textbf{hệ số phản xạ} (reflection coefficient) và \textbf{độ lợi truyền dẫn} (transmission gain)
    cho các đầu vào và đầu ra của mạng hai cổng. 
    Khái niệm cơ bản này xác định tham số S cho 
    mạng nhiều cổng, tính toán các tham số như 
    \textit{return loss}, \textit{VSWR} và 
    \textit{insertion loss}. 
    Trong bối cảnh này, thuật ngữ tham số S hay 
    tham số \textbf{tán xạ} đề cập đến cách dòng điện hoặc 
    điện áp di chuyển bị ảnh hưởng khi chúng gặp sự 
    cố trên đường truyền.
    
    \begin{figure}[h]
        \centering
        \includegraphics[width=0.5\textwidth]{figures/freq_plot.png}
        \caption{Frequency plot of absolute values $S_{11}$ scattering parameters equal to the return loss. Notice 0 dB at approximately 2 GHz.}
        \label{fig:freq_plot}
    \end{figure}

    \section{Định nghĩa S-parameters}
        Ma trận tham số S cho mạng 2 cổng là thành phần cơ bản để xây dựng các ma trận bậc cao hơn cho các mạng mở rộng hơn. 
        Mối quan hệ giữa sóng đi ra (sóng phản xạ) và sóng tới và ma trận tham số S có thể được biểu thị 
        trong công thức sau và nhân với nhau để có được các phương trình riêng cho $b_1$ và $b_2$.
        \cite{cadence2023sparams}
        \begin{equation}
            \begin{bmatrix}
                b_1 \\
                b_2
            \end{bmatrix}
                =
            \begin{bmatrix}
                S_{11} & S_{12} \\
                S_{21} & S_{22}
            \end{bmatrix}
            \begin{bmatrix}
                a_1 \\
                a_2
            \end{bmatrix}
        \end{equation}
        \begin{itemize}
            \item $a_i$: Sóng tới tại cổng $i$
            \item $b_i$: Sóng phản xạ tại cổng $i$
            \item $S_{ii}$: Hệ số phản xạ tại cổng $i$
            \item $S_{ij}$: Hệ số truyền dẫn từ cổng $i$ đến cổng $j$
        \end{itemize}

        \subsection{S-Parameters trên thang đo decibel (dB)}
            $S_{11}$ biểu thị tổn thất phản hồi của một thiết bị, 
            cho biết lượng công suất đầu vào được cung cấp cho thiết bị 
            phản xạ trở lại cổng đầu vào. 
            Lý tưởng nhất là không nên có công suất phản xạ và 
            100\% công suất phải được cung cấp cho thiết bị.\par

            Khi $S_{11}$ ở mức -10 dB, điều đó có nghĩa là ít nhất 90\% 
            công suất đầu vào được truyền hiệu quả đến thiết bị, 
            với ít hơn 10\% bị phản xạ.

        \subsection{S-Parameters trong ứng dụng antenna}
            Trong ứng dụng antenna, nếu $S_{11}$ là 0 dB, điều đó chỉ ra rằng
            toàn bộ công suất được phản xạ từ antenna và không bị bức xạ.
            Ngược lại khi $S_{11}$ là -10 dB, nghĩa rằng nếu 3 dB công suất được
            đưa vào antenna, thì chỉ có -7 dB bị phản xạ, phần còn lại được bức xạ 
            hoặc hấp thụ dưới dạng tổn thất trong antenna. Antenna được thiết kế
            để giảm thiểu tổn thất, đảm bảo hầu hết được truyền đến antenna đều được bức xạ.\cite{cadence2023sparams}

            Ví dụ, xem xét biểu đồ $S_{11}$ trong \autoref{fig:freq_plot}\footnote{Biểu đồ này thường được lấy bằng cách sử dụng Vector Network Analyzer (VNA), 
            có thể đo và hiển thị $S_{11}$.}, biểu đồ chỉ ra rằng bức xạ tối ưu của antenna xảy ra ở khoảng 2 GHz, tại đó $S_{11}$ đạt -12 dB. 
            Ngược lại, ở 1.2 GHz, antenna hầu như không phát ra bất kỳ công suất nào khi $S_{11}$ tiến gần đến 0 dB, biểu thị rằng hầu như 
            toàn bộ công suất đều bị phản xạ. Hơn nữa, băng thông của antenna có thể được xác định từ biểu đồ này. Nếu băng thông được 
            định nghĩa là dải tần số tại đó $S_{11}$ nhỏ hơn -6 dB, thì băng thông trải dài khoảng 1 GHz, với 2.5 GHz là giới hạn trên và 
            1.5 GHz là giới hạn dưới của băng tần.

    

    \section{$S_{11}$ (Return Loss)}
        Định nghĩa về return loss theo hệ số phản xạ đường truyền.
        $$RL = -10\log\left(\left|\frac{P_{ref}}{P_{fwd}}\right|\right) = -20\log\left(\left|\frac{V_{ref}}{V_{fwd}}\right|\right) = -20\log\left(\left|\Gamma\right|\right)$$
        \begin{itemize}
            \item $P_{ref}$: Công suất phản xạ
            \item $P_{fwd}$: Công suất tới
            \item $V_{ref}$: Điện áp phản xạ
            \item $V_{fwd}$: Điện áp tới
            \item $\Gamma$: Hệ số phản xạ
            \item $RL$: Tổn thất phản hồi (return loss)
        \end{itemize}

        $S_{11}$ được định nghĩa là suy hao phản hồi âm và do đó mang giá trị dB âm.
        $$S_{11} = -RL = 20\log\left(\left|\Gamma\right|\right)$$
        Trở kháng đầu vào và $S_{11}$ (return loss) đều liên quan đến hệ số phản xạ của đường truyền. Các tham số S thực là các hàm phức tạp của tần số và có thể có một tập hợp phức tạp các cộng hưởng/phản cộng hưởng (resonances/antiresonances); ví dụ về đường truyền được kết nối với điện dung tải 1 pF được kết thúc ở 50 Ohm được hiển thị bên dưới.\par
        \begin{figure}[h]
            \centering
            \includegraphics[width=0.5\textwidth]{figures/1pf_input.png}
            \caption{Comparison of $S_{11}$ and reflection coefficient at the input to a load component with 1 pF input capacitance.}
        \end{figure}

        Đường truyền hoạt động giống như một khoang cộng hưởng điển hình\footnote{\href{https://resources.pcb.cadence.com/blog/2019-what-is-a-cavity-resonator-and-how-is-one-used-in-pcb-design}{\color{blue}What is a Cavity Resonator and How is One Used in PCB Design}} và có cấu trúc cộng hưởng khi đường truyền rất ngắn. khi đường truyền dài hơn, tổn thất bắt đầu chiếm ưu thế và cộng hưởng trong phổ $S_{11}$ bắt đầu biến mất.\cite{cadence2021transmission}\par
        Khi kéo dài đường truyền ra vô cực, trở kháng đầu vào tại mỗi cổng giảm xuống (\autoref{eq:zin}). Đối với các đường truyền thực tế hoạt động ở tần số thực tế, cần phải mô tả hành vi tín hiệu theo trở kháng đầu vào và tham số S, đặc biệt là khi đường truyền ngắn.\cite{cadence2021transmission}\par

    \section{VSWR}
        Tỷ lệ sóng đứng điện áp (Voltage Standing Wave Ratio - VSWR) đo lường mức độ phù hợp của cổng, 
        tương tự như tổn thất phản hồi. Tuy nhiên, VSWR khác biệt ở chỗ nó là một tham số tuyến tính vô hướng, 
        biểu thị tỷ lệ giữa điện áp cực đại và cực tiểu của sóng đứng\footnote{\href{https://resources.pcb.cadence.com/blog/2022-understanding-standing-wave-patterns-on-interconnects-and-antennas}{\color{blue}Understanding Standing Wave Patterns on Interconnects and Antennas}}. 
        Do đó, VSWR được kết nối trực tiếp với độ lớn của hệ số phản xạ điện áp, 
        đến lượt nó, tương quan với $S_{11}$ cho cổng đầu vào hoặc $S_{22}$ cho cổng đầu ra.
        \begin{equation}
            {VSWR}_{IN} = \frac{1 + |S_{11}|}{1 - |S_{11}|}, \quad {VSWR}_{OUT} = \frac{1 + |S_{22}|}{1 - |S_{22}|}
        \end{equation}