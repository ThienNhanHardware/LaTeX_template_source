% ==========================
% Trang bìa
% ==========================

\thispagestyle{empty} % Xóa số trang trên trang bìa

\begin{titlepage}
    % ==========================
    % Thiết kế nền trang bìa
    % ==========================
    \begin{tikzpicture}[remember picture, overlay]
        % Vẽ hình chữ nhật màu xanh bên trái
        \shade[left color=black, right color=white!0, shading angle=37, opacity=0.75] 
            (current page.north west) rectangle 
            ([yshift=0.1\textheight]$(current page.south west) + (0.3\paperwidth,0)$);
        \shade[left color=TLgreen, right color=white!0, shading angle=37, opacity=0.75] 
            (current page.north west) rectangle 
            ([yshift=0.1\textheight]$(current page.south west) + (0.3\paperwidth,0)$);
        
        % Vẽ logo ở góc trên bên trái
        \node at ($(current page.north west) + (0.15\paperwidth,-0.25\paperheight)$) {
            \includegraphics[width=0.2\paperwidth, height=0.2\paperwidth, keepaspectratio]{figures/logo/logo_white.png}
        };
    \end{tikzpicture}
    
    % ==========================
    % Tiêu đề và tác giả
    % ==========================
    \vspace{0.1\paperheight} % Khoảng cách từ trên xuống
    \hspace{0.2\paperwidth} % Căn lề trái cho văn bản
    \begin{tabular}{p{0.7\textwidth}}
        \fontsize{40pt}{40pt}\selectfont\textbf{\docsName} \\ % Tên tài liệu
        \\ % Dòng trống
        \fontsize{20pt}{20pt}\selectfont \textit{\authNameFirst} \\ % Tác giả 1
        \fontsize{20pt}{20pt}\selectfont \textit{\authNameSecond} \\ % Tác giả 2
    \end{tabular}

    \vfill % Đẩy nội dung xuống cuối trang

    % Thông tin thành phố và ngày tháng
    \hspace{0.2\paperwidth}
    \begin{tabular}{p{0.75\textwidth}}
        \centering {\large \cityName, \today} % Thành phố và ngày tháng
    \end{tabular}
\end{titlepage}

% ==========================
% Thiết lập số trang sau trang bìa
% ==========================
\newpage
\setcounter{page}{2} % Đặt số trang bắt đầu từ 2